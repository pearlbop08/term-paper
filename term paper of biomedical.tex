\documentclass[11pt]{article}
\title{\Huge \textbf{Term Project}}
\usepackage{graphicx}
\graphicspath{{image/}}
\begin{document}
\maketitle
\centering On

\Huge "Mixed Reality in healthcare"
\setlength{\parskip}{0.5em}

\emph{\large Submitted by:}
\large Priyanka Harde

\large Roll No.:21111039
\begin{figure}[h]
\begin{center}
\includegraphics[scale=0.6]{nit.jpg}
\end{center}
\end{figure}

\textsc{\Large NATIONAL INSTITUTE OF TECHNOLOGY}

\large Under the supervision of : SAURABH GUPTA 
\clearpage 
\tableofcontents
\clearpage
\section{ ACKNOWLEDGEMENT}
 \raggedright I would like to express my sincere thanks and gratitude to Saurabh Sir for letting me work on this project. I am very grateful to him for his support and guidance in completing this project.

I am thankful to my parents as well. I was able to successfully complete this project with the help of their guidance and support. Finally, I want to thank all my dear friends as well.
\section{Abstract}


Mixed reality (MR) is an area of computer research dealing with the combination of real-world and computer-generated data (virtual reality), where computer-generated graphical objects are visually mixed into the real environment and vice versa in real time. This term paper contains an introduction to this modern technology in medical field .
\subsection{Keywords-}
virtual reality ,mixed reality,augmented reality,Hologram

\section{\textbf{INTRODUCTION}}
Immersive technologies, which includes augmented, virtual, and mixed reality(Milgram and Kishino, 1994) , are increasingly accessible to medical educators. As part of a larger project, we are interested in technologies to support simulation-based medical education which do more than present 3D information on an immobile 2D flatscreen monitor. Specifically, we identify how these technologies can improve the detection of subtle cues during medical diagnosis, which is an example of the macrocognition function of sensemaking (Patterson and Hoffman, 2012).



\section{\textbf{ What is MR?}}
Mixed reality is a blend of physical and digital worlds, unlocking natural and intuitive 3D human, computer, and environmental interactions. This new reality is based on advancements in computer vision, graphical processing, display technologies, input systems, and cloud computing. 



The term "mixed reality" was introduced in a 1994 paper by Paul Milgram and Fumio Kishino, "A Taxonomy of mixed reality Visual Displays." Their paper explored the concept of a virtuality continuum and the taxonomy of visual displays. Since then, the application of mixed reality has gone beyond displays to include:
\begin{figure}[h]
\includegraphics[scale=0.3]{mixedrealityspectrum-worlds.png}
\caption{Mixed Reality}
\end{figure}

Mixed reality is a blend of physical and digital worlds, unlocking natural and intuitive 3D human, computer, and environmental interactions. This new reality is based on advancements in computer vision, graphical processing, display technologies, input systems, and cloud computing. The term "mixed reality" was introduced in a 1994 paper by Paul Milgram and Fumio Kishino, "A Taxonomy of mixed reality Visual Displays." Their paper explored the concept of a virtuality continuum and the taxonomy of visual displays. Since then, the application of mixed reality has gone beyond displays to include:

\begin{itemize}
\item  Environmental understanding: spatial mapping and anchors.
\item  Human understanding: hand-tracking, eye-tracking, and speech input.
Spatial sound.
 \item Locations and positioning in both physical and virtual spaces.
\item  Collaboration on 3D assets in mixed reality spaces.
\item Environmental understanding: spatial mapping and anchors.
\item  Human understanding: hand-tracking, eye-tracking, and speech input.
Spatial sound.
\item  Locations and positioning in both physical and virtual spaces.
\item  Collaboration on 3D assets in mixed reality spaces.

\end{itemize}
\begin{figure}[h]
\centering \includegraphics[scale=0.3]{hololens 2.jpg}
\caption{hololens 2}
\end{figure}

\section{\textbf{How does it help in medical sector?}}
This new form of interactive tech is actually a mixture of virtual reality and augmented reality to create a new way to interact with the world around us. Also known as “hybrid reality” and “extended reality,” MR has the potential to change just about every industry you can think of Healthcare is no exception. Imagine a world where simple and complex medical procedures alike can be made easier and safer. That world is rapidly becoming reality-no pun. Let’s look at three ways mixed reality has the potential to revolutionize the healthcare industry.
\begin{itemize}




\item \large Instant diagnoses-
VR has been used for years to let doctors realistically visualize a patient’s anatomy that would otherwise be impossible to see. MR now presents healthcare professionals with real-time data and visuals that can change procedures forever and for the better. MR glasses are being built that can display images and information on top of a patient’s body and even perform instant analysis of a patient’s condition. The day is almost here when doctors wearing MR glasses will be able to look at a patient and instantly know not just a patient’s vitals, but what the most likely diagnoses are. 
\item \large Medical training, mixed reality style-

Typically, medical students can learn to do surgery in two ways: observing a live surgery or by operating on a corpse. Well, now there’s a third option: operating on a corpse using mixed reality. By leveraging MR, medical students can operate on cadavers in an environment that replicates what it’s like to perform surgery on a live patient. With the implementation of half reality and half simulation, these mixed reality surgeries can respond to a student’s actions as if it were a real procedure. Students can now practice with realistic virtual procedures that don’t risk anyone’s life.

\item \large Enhanced surgery-
Surgeons can make use of mixed reality and carry out virtual procedures with more accuracy than ever. Mixed reality glasses or screens can project real-time information (such as blood pressure and heart rate), patient imaging and more. MR can even help doctors monitor vitals and changes in a patient’s condition better than they can with the naked eye. MR imaging also provides real-time 3D views of anatomy, giving surgeons more detail and helping them make better informed decisions during procedures. As this technology progresses over time, even some of the riskiest procedures may become routine with the benefits MR offers. 



\end{itemize}
\section{\textbf{Difference between MR ,VR AND AR}}

Augmented reality (AR)- adds digital elements to a live view often by using the camera on a smartphone. Examples of augmented reality experiences include Snapchat lenses and the game Pokemon Go. 


Virtual reality (VR)- implies a complete immersion experience that shuts out the physical world. Using VR devices such as HTC Vive, Oculus Rift or Google Cardboard, users can be transported into a number of real-world and imagined environments such as the middle of a squawking penguin colony or even the back of a dragon.\begin{figure}[h]
\centering \includegraphics[scale=0.5]{Augmented-reality-and-virtual-reality-defined.jpg}
\caption{Differences between MR,VR AND AR}
\end{figure}

In a Mixed Reality (MR) -experience, which combines elements of both AR and VR, real-world and digital objects interact. Mixed reality technology is just now starting to take off with Microsoft’s HoloLens one of the most notable early mixed reality apparatuses.Using multiple sensors, advanced optics, and holographic processing that melds seamlessly with its environment. These holograms can be used to display information, blend with the real world, or even simulate a virtual world.




\subsection{Conclusion}
  Mixed reality research is progressing quite well, although it requires significant financial resources. On the other hand, this technology offers a very immersive experience for its users. Mixed reality allows to bring gaming, education, training, and presentation of various kinds of designs up to an entirely new level. It represents a new form of visualization of real objects, extended with virtual information. Models can be created using 3D modeling tools, including CAD software, and inserted to a real scene.
  
  
\subsection{References-}
docs.microsoft.com,www.channelfutures.com,www.gvsu.edu,
www.intechopen.com
 
\end{document}


